\documentclass[12pt,a4paper,oneside]{article}

\usepackage{graphicx}
\usepackage{enumerate}
\usepackage[shortlabels]{enumitem}

\setenumerate[1]{label=\thesection.\arabic*}

% Use Crimson for the body
\usepackage{crimson}

% Use FUCXEDCAPS for the section headings
\usepackage{fontspec}
\usepackage{titlesec}
\setsansfont[Ligatures=TeX]{[FUCXEDCAPSRegular.otf]}
\titleformat{\section}{\normalfont\sffamily\Large\bfseries}{\thesection}{1em}{}

\begin{document}

\begin{titlepage}
  \centering
  \includegraphics[scale=1.5]{XR-logo-Black-Stacked}\par
  \vspace{6em}
  {\normalfont\sffamily\huge XR UK Constitution v1 \par}
  \vfill
  {\large Produced: \today \par}
\end{titlepage}

\cleardoublepage

\tableofcontents

\cleardoublepage

\section*{Introduction}

\cleardoublepage

\section{XR UK Organism Structure}

\begin{enumerate}
\item
  The structure of XR UK is in units of teams (also sometimes called groups),
  which are clustered into circles of multiple teams.

\item
  Teams are autonomous \& self-governing. This means that it’s up to the teams
  how they do the work to achieve their mandate.

\item
  Each team has its own \textbf{mandate} which needs to have at least one, or
  any combination of the following:
  \begin{itemize}
  \item
    \textbf{purpose:} the reason the team exists/what it is working towards, as
    a smaller part of the purpose of the whole organism

  \item
    \textbf{accountabilities} which define the ongoing work that needs to be
    done by a team, what that team is accountable for, and expectations others
    can have of a team

  \item
    \textbf{domain} which conveys ownership. So for example, if a team owns the
    domain of the website, it means no other person or team can change the
    website without owning team’s permission. Could also apply to things like a
    Social Media account, contact list, bank account, script for a talk.
  \end{itemize}

\item
  The teams are in a structure where larger teams have sub-teams within them,
  and sub-teams can have their own sub-teams and so on, like Russian dolls. A
  sub-team is a \textbf{whole}, autonomous team in its own right, while also
  being \textbf{a part} of a larger team (this is the same structure as found in
  natural systems and is called a holarchy).

\item
  Each team should have one or two named coordinators (internal \& external)
  with mandates as described here, and: % TODO: Link this to the Appendix
  \begin{itemize}
  \item
    If a team is a sub-team of a larger team, the external coordinator is a
    member of both the broader team and sub-team.
  \item
    The external coordinator can nominate someone to attend the larger team in
    their place who represents the sub team as the external coordinator.
  \item
    This means that information and tensions can flow in both ways, both into
    and out of all teams and be addressed at the appropriate level of scale.
  \item
    The XR UK structure should be published \& updated so that everyone can see:
    \begin{itemize}
    \item the structure and how the teams fit together
    \item mandates for each team
    \item the 2 coordinators of each team
    \item a contact email address for each team
    \item any mandates for roles in a team, plus who is filling those roles
    \end{itemize}
  \end{itemize}

\item
  Projects emerge which don’t seem to fit into the structure: the circles aren’t
  boxes to be boxed in by! If a project arises which involves work of people
  from multiple circles, one person should take the lead on the project (ideally
  with a mandate which covers it) and then create an ad-hoc project team to do
  the work with people from wherever relevant. This team can autonomously
  self-organise and doesn’t need to be represented in the structure.

\item
  By default, all resources and strategies are owned and defined by the Anchor
  Circle unless specifically delegated to another role/team/circle.
\end{enumerate}

\section{Roles Within Teams}

\begin{enumerate}
\item
  Creating Internal Roles: it is recommended that teams define their own
  internal roles inside their teams with clear mandates for all ongoing work
  within a team. This increases clarity and reduces potential confusion about
  who is doing what.

\item
  When a team defines roles with mandates, the authority for what is included in
  the mandates is then devolved from the team level into a role. This means that
  whatever is in those mandates no longer gets decided at a team level, instead
  it gets decided by the role-holder.

\item
  Or teams may choose to not define any roles. This is the least preferable
  option since it reduces clarity about who will do what and has the potential
  for more confusion.

\item
  Filling Roles: Here’s 3 options to decide who fills which roles. It’s
  recommended to do them in this order:
  \begin{itemize}
  \item
    people volunteer for the roles they want to fill (authority is earned by
    taking action)

  \item
    if there are more volunteers than opportunities to fill roles there are two
    options for electing someone into a role:
    \begin{itemize}
    \item
      The Integrative election process which is used in Holacracy
    \item
      A simple majority vote
    \end{itemize}
  \end{itemize}

\item
  In some cases it’s appropriate for a single role to be filled by multiple
  people (e.g. meeting facilitator). In others it may not be.

\item
  In this system authority is earned; people earn the right to be a member of a
  team and attend meetings if they are actively filling a role and working on at
  least one project for that team as seen on a team’s project board (if using a
  project board). This prevents meetings being clogged up by people attending
  and sharing their views about how things should happen if they aren’t actually
  involved in any work.

\item
  Anyone can call for re-election of any coordinator role at any time.
\end{enumerate}

\end{document}